\documentclass{article}
\usepackage{graphicx} % Required for inserting images
\usepackage{float}
\usepackage{subcaption}
\usepackage[a4paper, total={7in, 10in}]{geometry}
\usepackage[dvipsnames]{xcolor}
\usepackage{amsmath}
\usepackage[unicode]{hyperref}
\usepackage{fancyhdr}

\title{Evaluasi Basic Rescue}
\author{Awwal Ainur Rizqi}
\date{23 \& 25 October 2025}

\begin{document}
\maketitle

\section{Introduksi}
Pengenalan latihan \textit{Basic Rescue} disajikan dengan mengenali dan memahami beberapa bagian berikut.
\begin{enumerate}
    \item \textcolor{OliveGreen}{Approaching}
    \item Releasing Victim
    \item Evac
\end{enumerate}
Berikut untuk tanggal, porsi, dan bagian dari latihan yang masih harus dijalani masih dalam teknik \textcolor{OliveGreen}{Approaching}. Adapun terkait dengan teknis dari latihan, dengan kasus:\\
\textcolor{OliveGreen}{[[\texttt{Korban menggunakan Descender dan korban di-\textit{descend} ke bawah}]]}.\\
Term untuk mempermudah pemahaman, dengan mengguanakan font \texttt{monospace}:
\begin{itemize}
    \item Untuk mempersingkat kata carabiner dibuat singkatan 'c.-\{tipe/alat lain/istilah lain\}';
    \item Penghubung Rescuer dengan korban menggunakan carabiner lock screw, dengan tambahan sling quickdraw, disesuaikan dengan rescuer, disebut sebagai '\texttt{penghubung}';
    \item Seluruh Teknik SRT dasar meliputi lintasan polos, deviasi, intermediate/rebelay, dan juga simpul dengan definisi yang mengacu pada konsep latihan ASC dan peserta mampu melampaui lintasan dan istilah dipakai semisal '\texttt{lintasan simpul}', maka yang diharapkan memahami dan mampu melewati lintasan simpul;
    \item Carabiner Descender dan friksi yang menjadi satu kesatuan, asumsi tidak menggunakan Carabiner \textit{Freino} (PETZL), apabila disatukan menjadi satu kesatuan, diistilahkan dengan '\texttt{c.des-frik}';
    \item Untuk peralatan \textit{ascender}, dalam hal ini \textit{chest ascender} dan \textit{hand ascender} diistilahkan '\texttt{ascender}', dan istilah \texttt{ascend} untuk istilah '\textit{ascending}';
    \item \texttt{descend} untuk istilah '\textit{descending}' menggunakan \textit{ascender} atau \textit{descender};
    \item Carabiner cowstail pendek dan/atau panjang, diistilahkan dengan '\texttt{cow.short}' dan '\texttt{cow.long}' secara berurutan.
\end{itemize}
Berikut kasus yang dihadapkan:
\begin{enumerate}
    \item Rescuer mendekati korban dari bawah, diusahakan mendapatkan posisi yang pas dengan mengatur panjang pendek untuk memasang \texttt{penghubung} pada descender korban;
    \begin{enumerate}
        \item Pemasangan \texttt{penghubung} didekatkan dengan korban dengan tujuan untuk mengontrol \textit{descender} korban. Jika agak jauh dari korban (\texttt{penghubung} terlalu panjang), akan menyulitkan dalam manuver untuk melepas dan kontrol \textit{descender} korban [\textcolor{NavyBlue}{catatan 1: Hindari kondisi ini, diusahakan posisi rescuer dekat dengan korban}];
        \item Kondisi \texttt{ascend} dengan posisi rescuer di bawah \textit{descender} korban, beberapa kondisi terlampir sebagai berikut.
        \begin{itemize}
            \item Jika dapat memasang \texttt{penghubung} pada \texttt{c.des-frik} korban, maka dapat dipasang langsung, dengan catatan \texttt{penghubung} tidak terlalu panjang, seperti yang telah dijelaskan di atas;
            \item Jika tidak, maka rescuer dapat menggunakan teknik melewati \texttt{lintasan simpul} dengan melihat \textit{descender} sebagai simpul dengan mengaitkan carabiner cowstail pendek pada maillon rapide (MR) korban [\textcolor{NavyBlue}{catatan 2: Pemasangan \texttt{c.cow.short} pada MR korban pada frame bawah dengan posisi gate \texttt{c.cow.short} menghadap rescuer}] sebagai backup sebelum memasang \texttt{penghubung};
        \end{itemize}
        \item Diharapkan kondisi kedua adalah kondisi yang paling ideal untuk memasang \texttt{penghubung} pada korban, karena dengan posisi ini, rescuer yang dekat atau hampir sejajar dengan korban akan memperbesar mobilitas rescuer, sehingga mempermudah dalam mengontrol pengaman korban;
        \item Penghubung telah terpasang, rescuer dapat melepas \texttt{ascender} kemudian, turun bersama korban dengan menggunakan \textit{descender} korban [\textcolor{NavyBlue}{catatan 3: Pada kondisi ini, manuver dari rescuer akan sedikit kesulitan dalam kontrol \textit{descender} korban, dikarenakan posisi descender agak sedikit jauh di atas, terbalik dari rescuer, sehingga jika \texttt{penghubung} agak panjang, semakin sulit bagi rescuer dalam kontrol \textit{descender} korban, oleh karena itu, untuk sementara, diakali dengan \textit{italian-hitch} diinstall pada c.friksi rescuer, tidak lupa untuk dikunci sebelum bermanuver melepas descender korban dan memindahkan tarikan beban pada c.friksi yang telah terinstall sebelumnya, diusahakan pelan untuk menghindari hentakan}].
    \end{enumerate}
    \item Rescuer mendekati korban dari atas, dikarenakan lintasan tali sudah terbebani dengan korban, maka lintasan yang dilalui rescuer merupakan lintasan tali tegang, oleh karena itu, diperlukan teknik khusus tambahan dengan beberapa kondisi rescuer dalam mendekati korban:
    \begin{enumerate}
        \item \textcolor{Emerald}{Mendekati korban dengan menggunakan \texttt{ascender}}:
        \begin{itemize}
            \item \texttt{descend} sampai dekat dengan korban;
            \item Memasang \texttt{penghubung} pada \texttt{c.des-frik} korban dan diusahakan dekat dengan korban;
            \item Kemudian melepas \texttt{ascender} sampai beban rescuer terbebani pada \texttt{c.des-frik} korban.
            \item Pada kondisi ini, dapat dilakukan manuver untuk melepas \textit{descender} korban dan \texttt{descend} ke bawah dengan mengingat pada [\textcolor{NavyBlue}{catatan 3}].
        \end{itemize}
        \item \textcolor{Purple}{Mendekati korban dengan menggunakan \textit{descender} --1}:
        \begin{itemize}
            \item Pemasangan \textit{descender} seperti yang ditunjukkan pada gambar:
            \begin{figure}[H]
                \centering
                \includegraphics[width=0.2\linewidth]{images/Asset 3.png}
                \caption{Instalasi \textit{descender} lintasan tali tegang versi 1 [Gambar belum dibuat]}
                \label{fig:descender1}
            \end{figure}
            \item Perhatikan terlebih dahulu ketegangan tali ketika hendak mendekat pada korban, istilahkan 'derajat kebebasan' bagi rescuer untuk manuver dalam melepas \textit{descender} [catatan: semakin dekat \textit{descender} rescuer dengan \textit{descender} korban, semakin kecil derajat kebebasan bagi rescuer];
            \item Kemudian diperhatikan untuk backup \textit{descender} menggunakan alat stopper, dapat berupa prusik yang dikaitkan pada ujung atas descender, dapat dikaitkan menggunakan carabiner, atau semisal stopper yang lebih mobile untuk mencegah terjadinya fall sebagai kontrol dalam menahan hentakan ketika hendak \texttt{descend};
            \item Jika rescuer telah mendapatkan posisi yang sesuai untuk mobilitas dalam melepas \textit{descender}, rescuer menggunakan kembali alat \texttt{ascender}-nya, kemudian menggunakan cara yang sama seperti pada kasus '(a) \textcolor{Emerald}{Mendekati korban dengan menggunakan ascender}', [\textcolor{NavyBlue}{catatan 4: Pemberlakuan ini ditujukan untuk memastikan kebiasaan bagi seorang rescuer ketika menghadapi medan berupa korban dengan lintasan jauh}].
        \end{itemize}
        \item \textcolor{RubineRed}{Mendekati korban dengan menggunakan \textit{descender} --2}:
        \begin{itemize}
            \item Pemasangan \textit{descender} seperti yang ditunjukkan pada gambar:
            \begin{figure}[H]
                \centering
                \includegraphics[width=0.2\linewidth]{images/Asset 3.png}
                \caption{Instalasi \textit{descender} lintasan tali tegang versi 2 \cite{Warild:1988}}
                \label{fig:descender2}
            \end{figure}
            \item Menggunakan cara yang sama seperti pada kasus '(b) \textcolor{purple}{Mendekati korban dengan menggunakan \textit{descender} --1}'
        \end{itemize}
    \end{enumerate}
\end{enumerate}

\section{Bagian Rigging}
Setup Rigging pada latihan:
    \begin{figure}[H]
    \begin{subfigure}{.5\textwidth}
      \centering
      \includegraphics[width=0.56\linewidth, angle=90]{images/RigginBasResc1.jpg}
      \caption{Rigging-man: Mas Dhiksa, Assisten: Mas Thomas [\textcolor{BrickRed}{23 Oktober 2025}]}
      \label{fig:rig123okt}
    \end{subfigure}\hspace{1cm}
    \begin{subfigure}{.5\textwidth}
      \centering
      \includegraphics[width=0.69\linewidth]{images/RigginBasRec2.jpg}
      \caption{Rigging-man: Awwal, Assisten: Mas Thomas [\textcolor{Magenta}{25 Oktober 2025}]}
      \label{fig:rig223okt}
    \end{subfigure}\hspace{2cm}
    \begin{subfigure}{.5\textwidth}
      \centering
      \includegraphics[width=0.5\linewidth]{images/RigginBasRec2a.jpg}
      \caption{detail pada (b) [\textcolor{Magenta}{25 Oktober 2025}]}
    \end{subfigure}
    \label{fig:rigging2325okt}
    \end{figure}
\section{Evaluasi Latihan}
\begin{enumerate}
    \item Pada posisi \textit{release} \textit{descender} korban, rescuer (Awwal) tidak menggunakan cara yang disarankan pada [\textcolor{NavyBlue}{catatan 3}], me-\textit{release} \textit{descender} korban secara langsung tanpa dibelokkan pada c.friksi, sehingga kesulitan untuk menimba tali. Kemudian dilakukan dengan membelokkan tali pada c.friksi rescuer tanpa ada \textit{hitch} atau \textit{stopper}, sehingga ketika hendak bermanuver pada c.friksi rescuer ada hentakan yang harus diterima sepanjang belokan tali pada tangan saat menimba;
    \item Perbendaharaan alat masih kurang rapih, dibutuhkan ketelitian dalam menggunakan alat, menghindari kondisi 'tali terbelit' atau \textit{blocking} pada alat lain tanpa disadari;
    \item Tidak mengikuti instruksi instruktur untuk mencoba latihan secara berturut untuk mendapat \textit{feel/sense} 'kapan harus melepas \textit{descender}\{rescuer\}' saat \texttt{descend} pada tali tegang, mencoba teknik '2.(a) \textcolor{Emerald}{Mendekati korban dengan menggunakan \texttt{ascender}}' agar tercapai posisi yang cocok oleh rescuer untuk melepas \textit{descender}-nya [\textcolor{BrickRed}{23 Oktober 2025}];
    \item Belum terbiasa dalam memasang dan menggunakan pengaman ketika hendak instalasi lintasan rigging (kasus memasang lintasan menggunakan tangga tinggi tanpa memasang pengaman tambahan). Ketika rigging:
    \begin{enumerate}
        \item Belum cukup menghemat waktu (faktor kecepatan) dalam memasang \textit{anchor} pada medan yang telah disediakan;
        \item Perlu adanya latihan terkait pemasangan \textit{tape/webbing} pada 'medan yang tidak dimungkinkan rusak/jebol, kalaupun ada fall, kemungkinan besar alatnya yang rusak', dengan tidak mengurangi prinsip orientasi \textit{main-backup anchor}, menggunakan sistem distribusi beban lebih diunggulkan (kasus kali ini, \textit{Y-anchor}), dengan pembuatan \textit{Y-anchor} yang tidak memberikan perubahan berarti pada lintasan dengan mengetahui medan yang dibebani sangat kuat dan tidak dimungkinkan untuk rusak/patah/jebol;
        \item Perlu memahami medan yang dihadapi (contoh penerapan lapangan, memahami pembentukan batuan sekitar gua), pengetahuan terkait kekuatan tape dan bentuk pemasangannya untuk menghindari terjadinya fall[\textcolor{Magenta}{25 Oktober 2025}].
    \end{enumerate}
\end{enumerate}
 Revisi dianjutkan pada latihan berikutnya.

\bibliographystyle{IEEEtran}
\bibliography{main}
\end{document}
